\documentclass[a4paper,11pt]{article}

%%%%%%%%%%%%%%%%%%%%%%%%%%%%%%%%%%%%%%%%%%%%%%%%%%%%%%%%%%%%%%%%%%%%%%%%
% Paquetes utilizados
%%%%%%%%%%%%%%%%%%%%%%%%%%%%%%%%%%%%%%%%%%%%%%%%%%%%%%%%%%%%%%%%%%%%%%%%

\usepackage[margin=0.8in]{geometry}

% Gráficos complejos
\usepackage{graphicx}
\usepackage{caption}
\usepackage{subcaption}
\usepackage{placeins}

% Soporte para el lenguaje español
\usepackage{textcomp}
\usepackage[utf8]{inputenc}
\usepackage[T1]{fontenc}
\DeclareUnicodeCharacter{B0}{\textdegree}
\usepackage[spanish]{babel}

% Código fuente embebido
\usepackage{listings}
\usepackage{courier}

% PDFs embebidos para el apéndice
\usepackage{pdfpages}

% Matemáticos
\usepackage{amssymb,amsmath}

% Tablas complejas
\usepackage{multirow}

% Formato de párrafo
\setlength{\parskip}{1ex plus 0.5ex minus 0.2ex}

% Subrayado de palabras
\usepackage[normalem]{ulem}

% Agregados en el documento
\usepackage{array}
\usepackage{fancyhdr}
\renewcommand{\headrulewidth}{0pt}
\renewcommand{\footrulewidth}{0pt}

% Formato de listados de código
\lstset{
  basicstyle=\footnotesize\ttfamily,
  numberstyle=\tiny,
  numbersep=5pt,
  tabsize=2,
  extendedchars=true,
  breaklines=true,
  stringstyle=\color{white}\ttfamily,
  showspaces=false,
  showtabs=false,
  xleftmargin=17pt,
  framexleftmargin=17pt,
  framexrightmargin=5pt,
  framexbottommargin=4pt,
  showstringspaces=false,
  language=SQL
}
\usepackage{caption}
\DeclareCaptionFont{white}{\color{white}}
\DeclareCaptionFormat{listing}{\colorbox[cmyk]{0.43, 0.35, 0.35,0.01}{\parbox{\textwidth}{\hspace{15pt}#1#2#3}}}
\captionsetup[lstlisting]{format=listing,labelfont=white,textfont=white, singlelinecheck=false, margin=0pt, font={bf,footnotesize}}

%%%%%%%%%%%%%%%%%%%%%%%%%%%%%%%%%%%%%%%%%%%%%%%%%%%%%%%%%%%%%%%%%%%%%%%%
% Título
%%%%%%%%%%%%%%%%%%%%%%%%%%%%%%%%%%%%%%%%%%%%%%%%%%%%%%%%%%%%%%%%%%%%%%%%

% Título principal del documento.
\title{\textbf{Trabajo Práctico: Agenda Médica}}

% Información sobre los autores.
\author{
  Celeste Maldonado,	\textit{P. 85.630},	\textit{maldonado.celeste@gmail.com}	\\
  Gisela Daye,		\textit{P. 87.602},	\textit{gisedaye@gmail.com}		\\
  Sergio Matías Piano,	\textit{P. 85.191},	\textit{smpiano@gmail.com}		\\
  Vanesa Conte,		\textit{P. 82.997},	\textit{vanexius@gmail.com}		\\
  \\
  \normalsize{Grupo 19}							\\
  \normalsize{Docente: Luis Fulco}					\\
  \normalsize{1er. Cuatrimestre de 2014}                           	\\
  \normalsize{75.15 - Bases de datos}                              	\\
  \normalsize{Facultad de Ingeniería, Universidad de Buenos Aires}
}
\date{}

%%%%%%%%%%%%%%%%%%%%%%%%%%%%%%%%%%%%%%%%%%%%%%%%%%%%%%%%%%%%%%%%%%%%%%%%
% Documento
%%%%%%%%%%%%%%%%%%%%%%%%%%%%%%%%%%%%%%%%%%%%%%%%%%%%%%%%%%%%%%%%%%%%%%%%

\begin{document}

% ----------------------------------------------------------------------
% Top matter
% ----------------------------------------------------------------------
\thispagestyle{empty}
\maketitle

\begin{abstract}

  Este informe resume el desarrollo del trabajo práctico de la materia Base
  de Datos (75.15) dictada en el primer cuatrimestre de 2014 en la Facultad de
  Ingeniería de la Universidad de Buenos Aires. El mismo consiste en el
  modelado de datos de un software de agenda médica,
  cuyos requisitos fueron extraído de un caso de estudio real.

\end{abstract}

\clearpage

% ----------------------------------------------------------------------
% Tabla de contenidos
% ----------------------------------------------------------------------
\tableofcontents
\clearpage


% ----------------------------------------------------------------------
% Desarrollo
% ----------------------------------------------------------------------
\part{Desarrollo}


\section{Modelo de entidad-interrelación} \label{sec:der}

\subsection{Hipótesis}

\begin{enumerate}
  \item demanda espontanea: El paciente se presenta fisicamente y pide el turno el mismo día.

\item Recibir solicitud de recurso: los turnos los solicita el paciente.

\item Los hospitales o clínicas aceptan todas las obras sociales.

\item Los intervalos de tiempo se diferencian por especialidad y no por subespecialidad del profesional.

\item Se reserva al menos 2 consultas por día para demanda espontanea.
      Éstos 2 turnos se van a encontrar al finalizar la agenda.

\item Siempre que se da de alta una Especialidad clasifica en una subespecialidad.

\item Suponemos que los bloques tienen sobreturno para poder darle la flexibilidad al profesional 
      que decida que días acepta hacer sobreturnos.

\item Cuando un profesional anula un bloque de horas, las instancias de turnos creadas para dicho bloque, 
      deben ser modificadas. En tal caso la recepcionista llama a cada paciente para reasignar el turno.
      No se altera el atributo de anulado sobre el turno.

\item Los turnos solamente se asignan por teléfono y personalmente.

\item Los turnos se asignan solamente por Mesa de Turno.

\item Tipo de turno se considera a primera vez, visita subsiguiente, demanda espontánea, cualquier otro 
      tipo que determine profesional, especialidad o servicio.

\item Los turnos se asignan por profesional, especialidad o servicio.

\item Los turnos son solicitados por pacientes.

\item Los servicios, quirófanos y camas son solicitados por médicos. No existe el sobreturno en éste caso.

DUDAS?
\item Puede existir un mismo domicilio asociado a varios pacientes o profesionales.

\item Los turnos se generarán a partir de los block horas que no se encuentren 
bloqueados al momento de crear la agenda.

\item Cada profesional está asociado a por lo menos una subespecialidad. No existen 
profesionales que no esten asociados a alguna subespecialidad.

\item Por cada block de hora existe una sola forma de atención a pacientes y un 
solo tipo de turno.

\end{enumerate}



\subsection{Diagrama de entidad-interrelación}

 En la figura \ref{fig:der} se incluye el diagrama de entidad-interrelación
 final desarrollado para representar el dominio modelado cuyo relevamiento se
 detalla en el enunciado.

\begin{figure}[h!t]
  \centering
  \includegraphics[width=1.4\textwidth, angle=90]{build/images/der.jpeg}
  \caption{Diagrama de entidad-interrelación} \label{fig:der}
\end{figure}

\FloatBarrier

\section{\textbf{Dependencias}}
\begin{enumerate}
\item Existe una dependencia de existencia de la entidad Turno con la entidad Block 
hora.

\item Existe una dependencia de identidad entre las entidades Plan y Obra social. 
Plan es una identidad débil, para poder identificarla utilizamos la razón social 
de la Obra Social (que es clave de la entidad fuerte) y como atributo discriminante 
el nombre del plan.

\item Existe una dependencia de identidad entre las entidades Especialidad y Subespecialidad. 
Subespecialidad es una identidad débil, para poder identificarla utilizamos el 
código de la Especialidad (que es clave de la entidad fuerte) y como atributo 
discriminante la descripción de la Subespecialidad.

\item Existe una dependencia de identidad entre las entidades Turno y Comprobante. 
Comprobante es una identidad débil, para poder identificarla utilizamos el código 
del Turno (que es clave de la entidad fuerte) y como atributo discriminante el 
número de Comprobante.\pagebreak{}
\end{enumerate}
\section{\textbf{Diccionario de datos}}

\subsection{\textbf{Entidades\label{HToc293405806}}}

\subsubsection{\textbf{Paciente}}

\textit{Definición}

Persona que recibe atención médica en el hospital. 

\textit{Especificación de atributos}

• Número de documento: número de documento del paciente.

• Tipo de documento: tipo de documento presentado por el paciente. Puede ser 
LE, LC, DNI, PASAPORTE.

• Número de afiliado: número asignado al paciente dentro del hospital.

• Nombre y apellido: nombre y apellido del paciente.

• Fecha de nacimiento: día, mes y año en que nació el afiliado.

• Lugar de nacimiento: ciudad o pueblo y país donde nació el paciente.

• Tipo beneficiario: indica la condición del paciente frente al hospital. Puede 
ser titular o adicional.

• Teléfono: número de teléfono del paciente.

• Email: dirección de correo electrónico del paciente.

• Condición IVA: condición del paciente frente al impuesto al valor agregado 
definido por la AFIP.

• Condición paciente: estado del paciente.

• Número de historia clínica: referencia el número de historia clínica que 
contiene tratamientos y diagnósticos del paciente.

\textit{Especificación de identificador único}

• Tipo documento

• DNI\label{HToc293405807}

\subsubsection{\textbf{Domicilio}}

\textit{Definición}

Lugar donde un PACIENTE o un PROFESIONAL tiene residencia.

\textit{Especificación de atributos}

• Código de domicilio: número entero que identifica al domicilio.

• Localidad: lugar perteneciente a una PROVINCIA.

• Provincia: subdivisión administrativa de un país.

• Calle: nombre de la calle correspondiente al domicilio.

• Número: número que identifica al terreno correspondiente al domicilio.

• Piso: nivel correspondiente al domicilio, en caso de que se trate de un edificio.

\textit{Especificación de identificador único}

• Código de domicilio\label{HToc293405808}

\subsubsection{\textbf{Profesional}}

\textit{Definición}

Un médico del hospital que diagnostica, atiende o trata a un PACIENTE.

\textit{Especificación de atributos}

• Nombre y apellido: nombre y apellido del profesional.

• Número de documento: número de documento del profesional.

• Tipo de documento: tipo de documento presentado por el profesional. Puede ser 
LE, LC, DNI, PASAPORTE.

• Celular: número de celular del profesional.

• Email: dirección de correo electrónico del profesional.

\textit{Especificación de identificador único}

• Tipo de documento

• Número de documento\label{HToc293405809}

\subsubsection{\textbf{Especialidad}}

\textit{Definición}

área de la medicina en que se especializa un PROFESIONAL.

\textit{Especificación de atributos}

• Código de especialidad: número entero que identifica la especialidad.

• Descripción: nombre de la especialidad.

\textit{Especificación de identificador único}

• Código de especialidad\label{HToc293405810}

\subsubsection{\textbf{Subespecialidad}}

\textit{Definición}

área dentro de la especialidad en que se especializa un PROFESIONAL. Depende de 
la especialidad.

\textit{Especificación de atributos}

• Descripción: nombre de la subespecialidad.

\textit{Especificación de identificador único}

• Código de especialidad

• Descripción\label{HToc293405811}

\subsubsection{\textbf{Acción médica}}

\textit{Definición}

Todo procedimiento médico, servicio, estudio o tratamiento brindado por el hospital. 
Incluye servicio de quirófano y camas.

\textit{Especificación de atributos}

• Código de acción médica: número entero que identifica la acción médica.

• Unidades: cantidad de unidades de tiempo asignadas a la acción médica. La 
unidad es 5 minutos.

• Descripción: denominación de la acción médica.

• Condiciones paciente: Texto de las condiciones médicas en la cual el paciente 
debe presentarse para realizarse el procedimiento médico a aplicar

\textit{Especificación de identificador único}

• Código de acción médica\label{HToc293405812}

\subsubsection{\textbf{Turno}}

\textit{Definición}

Fecha y hora en que se cita a un paciente que debe ser atendido en el hospital.

\textit{Especificación de atributos}

• Código de turno: número entero que identifica al turno.

• Fecha turno: día, mes y año en que se cita al paciente.

• Hora turno: hora y minutos en que el paciente fue citado.

• Sobreturno: atributo que puede valer si o no según el turno sea un sobreturno 
o no lo sea.

• Bloquado: atributo que puede valer si o no según el turno se encuentre bloqueado 
o no por el profesional asignado al turno.

• Lugar: dónde se atenderá al paciente citado.

\textit{Especificación de identificador único}

• Código de turno\label{HToc293405813}

\subsubsection{\textbf{Plan}}

\textit{Definición}

Prestaciones y servicios que ofrece una obra social determinada.

\textit{Especificación de atributos}

• Nombre: denominación del plan

\textit{Especificación de identificador único}

• Razón social de la obra social asociada

• Nombre del plan\label{HToc293405814}

\subsubsection{\textbf{Obra social}}

\textit{Definición}

Entidad que recibe aportes de un paciente y la brinda cobertura.

\textit{Especificación de atributos}

• Tipo de entidad: si la obra social es prepaga o pertenece a un sindicato.

• Razón social: denominacion legal de la obra social

\textit{Especificación de identificador único}

• Razón social\label{HToc293405815}

\subsubsection{\textbf{Block horas}}

\textit{Definición}

Designa los días y horarios en que un PROFESIONAL estará disponible para atender 
PACIENTES.

\textit{Especificación de atributos}

• Código de block de hora: número entero que identifica al block de horas.

• Hora desde: hora en que el PROFESIONAL comienza a atender PACIENTES.

• Hora desde: hora en que el PROFESIONAL finaliza la atención de PACIENTES.

• Día: día de la semana en que atiende el PROFESIONAL.

• Semana: semana del año en que es válido el block.

• Tipo de turno: indica características de los turnos que se generan a partir 
del block. Puede ser primera vez, visita subsiguiente, demanda espontánea.

• Tipo de agenda: indica la modalidad de atención de PACIENTES. Puede ser por 
orden de llegada o sin orden en particular.

• Bloqueado: permite indicar si un block está bloqueado por un PROFESIONAL o 
no.

• Acepta sobreturnos: vale sí si el PROFESIONAL decide aceptar sobreturnos dentro 
del block de horas, no si no.

• Acepta turnos fuera de agenda: toma el valor sí si el PROFESIONAL aceptará 
turnos fuera de agenda durante el horario del block, no, si no.

• Cantidad de pacientes: número de pacientes que el PROFESIONAL atenderá durante 
el tiempo que dure el block de horas.

\textit{Especificación de identificador único}

• Código de block de hora\label{HToc293405816}

\subsubsection{\textbf{Comprobante}}

\textit{Definición}

Contiene información relacionada con el turno que se conoce al momento que un 
PACIENTE reserva un turno.

\textit{Especificación de atributos}

• Número de comprobante: número entero que identifica el comprobante asociado 
al turno.

• Fecha de alta: fecha de alta del turno.

• Usuario: nombre del usuario que dio de alta el turno en el sistema.

\textit{Especificación de identificador único}

• Número de comprobante\label{HToc293405817}

\subsection{\textbf{Interrelaciones\label{HToc293405818}}}

\subsubsection{\textbf{Vive en}}

\textit{Definición}

Relaciona a un PROFESIONAL o un PACIENTE con el DOMICILIO donde reside.

\textit{Especificación de atributos}

-

\textit{Especificación de identificador único}

• Tipo de documento (PACIENTE)

• Número de documento (PACIENTE)

• Tipo de documento (PROFESIONAL)

• Número de documento (PROFESIONAL)\label{HToc293405819}

\subsubsection{\textbf{Se atiende}}

\textit{Definición}

Asocia a un PACIENTE y al TURNO que le ha sido asignado.

\textit{Especificación de atributos}

-

\textit{Especificación de identificador único}

• Tipo de documento

• Número de documento

• Código de turno\label{HToc293405820}

\subsubsection{\textbf{Asignado a}}

\textit{Definición}

Relaciona al PROFESIONAL con los TURNOS que le hayan sido asignados.

\textit{Especificación de atributos}

-

\textit{Especificación de identificador único}

• Código de turno\label{HToc293405821}

\subsubsection{\textbf{Se realiza}}

\textit{Definición}

Indica qué ACCIÓN MÉDICA se realizará en un TURNO.

\textit{Especificación de atributos}

-

\textit{Especificación de identificador único}

• Código de turno\label{HToc293405822}

\subsubsection{\textbf{Cubierto por}}

\textit{Definición}

Indica qué PLAN de OBRA SOCIAL cubre a un PACIENTE.

\textit{Especificación de atributos}

-

\textit{Especificación de identificador único}

• Tipo de documento

• Número de documento\label{HToc293405823}

\subsubsection{\textbf{Convenio}}

\textit{Definición}

Indica las caracterísiticas del convenio que tiene la el hospital con el PLAN 
de una OBRA SOCIAL determinada en relación a una ACCIÓN MÉDICA.

\textit{Especificación de atributos}

• Documentación requerida: Texto de la documentación del convenio, en la cual 
indica lo que el paciente debe presentar para realizarse el procedimiento médico 
a aplicar.

• Monto: cantidad de dinero aportado en el plan para la ACCIÓN MÉDICA. Es calculable.

• Porcentaje de exención: indica qué porcentaje del total del monto requerido 
para la ACCIÓN MÉDICA es cubierto por el PLAN.

\textit{Especificación de identificador único}

• Código de acción médica

• Razón social

• Nombre plan\label{HToc293405824}

\subsubsection{\textbf{Autorización}}

\textit{Definición}

Indica qué ACCIONes MÉDICAS un PROFESIONAL está autorizado a realizar.

\textit{Especificación de atributos}

-

\textit{Especificación de identificador único}

• Código de acción médica

• Tipo de documento

• Número de documento\label{HToc293405825}

\subsubsection{\textbf{Realizada por}}

\textit{Definición}

Indica qué SUBESPECIALIDAD realizará una ACCIÓN MEDICA.

\textit{Especificación de atributos}

-

\textit{Especificación de identificador único}

• Código de acción médica

• Código de especialidad

• Descripción de subespecialidad\label{HToc293405826}

\subsubsection{\textbf{Tiene}}

\textit{Definición}

Indica qué SUBESPECIALIDADes tiene un PROFESIONAL del hospital.

\textit{Especificación de atributos}

\textit{Especificación de identificador único}

• Código de especialidad

• Descripción de subespecialidad

• Tipo de documento

• Número de documento\label{HToc293405827}

\subsubsection{\textbf{Prevalece}}

\textit{Definición}

Permite modelar el requerimiento de que un PROFESIONAL solo puede tener una ESPECIALIDAD 
que prevalezca.

\textit{Especificación de atributos}

-

\textit{Especificación de identificador único}

• Tipo de documento

• Número de documento\label{HToc293405828}

\subsubsection{\textbf{Comprende}}

\textit{Definición}

Indica los TURNOs que componen un BLOCK HORA.

\textit{Especificación de atributos}

-

\textit{Especificación de identificador único}

• Código de turno\label{HToc293405829}

\subsubsection{\textbf{Define}}

\textit{Definición}

Indica a qué PROFESIONAL, desempeñando qué SUBESPECIALIDAD definió un BLOCK 
HORAS determinado.

\textit{Especificación de atributos}

-

\textit{Especificación de identificador único}

• Tipo de documento (PACIENTE)

• Número de documento (PACIENTE)

• Tipo de documento (PROFESIONAL)

• Número de documento (PROFESIONAL)

• Código de block horas\pagebreak{}\label{HToc293405830}

\section{\textbf{Modelo relacional\label{HToc293405831}}}

\subsection{\textbf{Diseño del modelo\label{HToc293405832}}}

\subsubsection{\textbf{Domicilio}}

DOMICILIO(\emph{codigo\_domicilio}, localidad, número, provincia, calle, piso);
\begin{itemize}
\item PK: (codigo\_domicilio)

\item FK: -

\item CC:

o (codigo\_domicilio)

\item ATRIBUTOS CON VALORES NULOS: -\label{HToc293405833}
\end{itemize}

\subsubsection{\textbf{Obra social}}

OBRA\_SOCIAL(\emph{razon\_social}, tipo\_entidad);

\begin{itemize}
\item PK: (razon social)

\item FK: -

\item CC:

o (razon\_social)

\item ATRIBUTOS CON VALORES NULOS: -\label{HToc293405834}
\end{itemize}

\subsubsection{\textbf{Plan}}

PLAN(\emph{razon\_social}, nombre\_plan);

\begin{itemize}
\item PK: (razon social, nombre\_plan)

\item FK: (razon\_social)

\item CC:

o (razon social, nombre\_plan)

\item ATRIBUTOS CON VALORES NULOS: -\label{HToc293405835}
\end{itemize}

\subsubsection{\textbf{Paciente}}

PACIENTE(\emph{tipo\_doc\_paciente, numero\_doc\_paciente}, numero\_afiliado, fecha\_nacimiento, 
lugar\_nacimiento, tipo\_beneficiario, telefono, email, condicion\_iva, condicion\_paciente, 
numero\_historia\_clinica, apellido\_nombre, codigo\_domicilio, razon\_social, 
 nombre\_plan);

\begin{itemize}
\item PK: (tipo\_doc\_paciente, numero\_doc\_paciente)

\item FK: 

o (codigo\_domicilio) 

o (razon\_social , nombre\_plan)

\item CC:

o (tipo\_doc\_paciente, numero\_doc\_paciente)

o (razon\_social, numero\_afiliado)

\item ATRIBUTOS CON VALORES NULOS:  fecha\_nacimiento, lugar\_nacimiento, tipo\_beneficiario, 
email, condicion\_iva, condicion\_paciente\label{HToc293405836}
\end{itemize}

\subsubsection{\textbf{Acción médica}}

ACCION\_MEDICA(\emph{codigo\_accion\_medica}, unidades, condiciones\_paciente, 
descripción);

\begin{itemize}
\item PK: (codigo\_accion\_medica)

\item FK:-

\item CC:

o (codigo\_accion\_medica)

\item ATRIBUTOS CON VALORES NULOS:  condiciones\_paciente.\label{HToc293405837}
\end{itemize}

\subsubsection{\textbf{Especialidad}}

ESPECIALIDAD(\emph{codigo\_especialidad}, descripción);

\begin{itemize}
\item PK: (codigo\_especialidad)

\item FK:-

\item CC:

o (codigo\_especialidad)

\item ATRIBUTOS CON VALORES NULOS: - \label{HToc293405838}
\end{itemize}

\subsubsection{\textbf{Subespecialidad}}

SUBESPECIALIDAD(\emph{codigo\_especialidad}, descripcion\_subespecialidad);

\begin{itemize}
\item PK: (codigo\_especialidad, descripcion\_subespecialidad)

\item FK: (codigo\_especialidad)

\item CC:

o (codigo\_especialidad, descripcion\_subespecialidad)

\item ATRIBUTOS CON VALORES NULOS: -\label{HToc293405839}
\end{itemize}

\subsubsection{\textbf{Profesional}}

PROFESIONAL(\emph{tipo\_doc\_profesional, numero\_doc\_profesional}, apellido\_nombre, 
email, número\_matrícula, celular,\textit{\textbf{ }}codigo\_domicilio, codigo\_especialidad)

\begin{itemize}
\item PK: (tipo\_doc\_profesional, numero\_doc\_profesional)

\item FK: 

o (codigo\_domicilio)

o (código\_especialidad)

\item CC:

o (tipo\_doc\_profesional, numero\_doc\_profesional) 

\item ATRIBUTOS CON VALORES NULOS:  email.\label{HToc293405840}
\end{itemize}

\subsubsection{\textbf{Block horas}}

BLOCK\_HORAS(\emph{codigo\_block\_horas}, día, semana, hora\_desde,hora\_hasta, 
acepta\_sobreturnos, acepta\_fuera\_agenda, bloqueado, tipo\_turno, tipo\_agenda, 
cantidad\_pacientes, unidades,\emph{ }tipo\_doc\_profesional, numero\_doc\_profesional, 
codigo\_especialidad, descripcion\_subespecialidad);

\begin{itemize}
\item PK: (codigo\_block\_horas)

\item FK: 

o (tipo\_doc\_profesional, numero\_doc\_profesional, codigo\_especialidad, descripcion\_subespecialidad)

\item CC:

o (codigo\_block\_horas)

\item ATRIBUTOS CON VALORES NULOS: acepta\_sobreturnos, acepta\_fuera\_agenda, bloqueado, 
tipo\_turno, tipo\_agenda, cantidad\_pacientes, unidades.\label{HToc293405841}
\end{itemize}

\subsubsection{\textbf{Turno}}

TURNO(\emph{código\_turno}, fecha\_turno, hora\_turno, observaciones, lugar, sobreturno, 
bloqueado, código\_block\_horas,código\_acción\_medica);

\begin{itemize}
\item PK: (codigo\_turno)

\item FK: 

o (codigo\_block\_horas)

o (codigo\_accion\_medica)

\item CC:

o (codigo\_turno)

\item ATRIBUTOS CON VALORES NULOS:  observaciones, lugar.\label{HToc293405842}
\end{itemize}

\subsubsection{\textbf{Comprobante}}

COMPROBANTE(\emph{código\_turno}, número\_comprobante, fecha\_alta, usuario);

\begin{itemize}
\item PK: (codigo\_turno, numero\_comprobante)

\item FK: 

o (codigo\_turno)

\item CC:

o (codigo\_turno, numero\_comprobante)

\item ATRIBUTOS CON VALORES NULOS:  fecha alta, usuario.
\end{itemize}

\textbf{Vive en (profesional)}

\textit{Diseño 1}

VIVE\_EN\_PROFESIONAL(\emph{ tipo\_doc\_profesional, numero\_doc\_profesional, 
}codigo\_domicilio)

\begin{itemize}
\item PK: (tipo\_doc\_profesional, numero\_doc\_profesional)

\item FK: 

o (codigo\_domicilio)

o (tipo\_doc\_profesional, numero\_doc\_profesional)

\item CC:

o (tipo\_doc\_profesional, numero\_doc\_profesional)

\item ATRIBUTOS CON VALORES NULOS: -
\end{itemize}

DOMICILIO(\emph{codigo\_domicilio}, localidad, número, provincia, calle, piso)

PROFESIONAL(\emph{tipo\_doc\_profesional, numero\_doc\_profesional}, apellido\_nombre, 
email, número\_matrícula, celular, codigo\_especialidad)

\textit{Diseño 2}

DOMICILIO(\emph{codigo\_domicilio}, localidad, número, provincia, calle, piso)

PROFESIONAL(\emph{tipo\_doc\_profesional, numero\_doc\_profesional}, apellido\_nombre, 
email, número\_matrícula, celular,\textit{\textbf{ }}codigo\_domicilio, codigo\_especialidad)

Se optó por el diseño 2 porque reduce el número de tablas.\label{HToc293405843}

\subsubsection{\textbf{Vive en (paciente)}}

\textit{Diseño 1}

VIVE\_EN\_PACIENTE(\emph{tipo\_doc\_paciente, numero\_doc\_paciente,} codigo\_domicilio)

\begin{itemize}
\item PK: (tipo\_doc\_paciente, numero\_doc\_paciente)

\item FK: 

o (codigo\_domicilio)

o (tipo\_doc\_paciente, numero\_doc\_paciente)

\item CC:

o (tipo\_doc\_paciente, numero\_doc\_paciente)

\item ATRIBUTOS CON VALORES NULOS: -
\end{itemize}

DOMICILIO(\emph{codigo\_domicilio}, localidad, número, provincia, calle, piso);

PACIENTE(\emph{tipo\_doc\_paciente, numero\_doc\_paciente}, numero\_afiliado, fecha\_nacimiento, 
lugar\_nacimiento, tipo\_beneficiario, telefono, email, condicion\_iva, condicion\_paciente, 
numero\_historia\_clinica, apellido\_nombre, razon\_social,  nombre\_plan);

\textit{Diseño 2}

DOMICILIO(\emph{codigo\_domicilio}, localidad, número, provincia, calle, piso);

PACIENTE(\emph{tipo\_doc\_paciente, numero\_doc\_paciente}, numero\_afiliado, fecha\_nacimiento, 
lugar\_nacimiento, tipo\_beneficiario, telefono, email, condicion\_iva, condicion\_paciente, 
numero\_historia\_clinica, apellido\_nombre, codigo\_domicilio, razon\_social, 
 nombre\_plan);

Se optó por el diseño 2 porque reduce el número de tablas.\label{HToc293405844}

\subsubsection{\textbf{Tiene\_Especialidad}}

TIENE\_ESPECIALIDAD(\emph{tipo\_doc\_profesional, numero\_doc\_profesional, codigo\_especialidad,} 
\emph{descripcion\_subespecialidad});

\begin{itemize}
\item PK: (tipo\_doc\_profesional, numero\_doc\_profesional, codigo\_especialidad, 
descripcion\_subespecialidad)

\item FK: 

o (tipo\_doc\_profesional, numero\_doc\_profesional)

o (codigo\_especialidad, descripcion\_subespecialidad)

\item CC:

o (tipo\_doc\_profesional, numero\_doc\_profesional, codigo\_especialidad, descripcion\_subespecialidad)

\item ATRIBUTOS CON VALORES NULOS:  -.
\end{itemize}

PROFESIONAL(\emph{tipo\_doc\_profesional, numero\_doc\_profesional}, apellido\_nombre, 
email, número\_matrícula, celular,\textit{\textbf{ }}codigo\_domicilio, codigo\_especialidad)

SUBESPECIALIDAD(\emph{codigo\_especialidad, }descripcion\_subespecialidad);\label{HToc293405845}

\subsubsection{\textbf{Prevalece}}

\textit{Diseño 1}

PREVALECE(\emph{tipo\_doc\_profesional, numero\_doc\_profesional}\textit{, }codigo\_especialidad);

\begin{itemize}
\item PK: (tipo\_doc\_profesional, numero\_doc\_profesional)

\item FK: 

o (tipo\_doc\_profesional, numero\_doc\_profesional)

o (codigo\_especialidad)

\item CC:

o (tipo\_doc\_profesional, numero\_doc\_profesional)

\item ATRIBUTOS CON VALORES NULOS:  -.
\end{itemize}

PROFESIONAL(\emph{tipo\_doc\_profesional, numero\_doc\_profesional}, apellido\_nombre, 
email, número\_matrícula, celular\textit{\textbf{, }}codigo\_domicilio);

ESPECIALIDAD(\emph{codigo\_especialidad}, descripción);

\textit{Diseño 2}

PROFESIONAL(\emph{tipo\_doc\_profesional, numero\_doc\_profesional}, apellido\_nombre, 
email, número\_matrícula, celular\textit{\textbf{, }}codigo\_domicilio,\textit{\textbf{ 
}}codigo\_especialidad);

ESPECIALIDAD(\emph{codigo\_especialidad}, descripción);

Se optó por el diseño 2 porque reduce el número de tablas.\label{HToc293405846}

\subsubsection{\textbf{Convenio}}

CONVENIO(\emph{codigo\_accion\_medica, razon\_social , nombre\_plan}, porcentaje\_exencion, 
documentación\_requerida, monto);

\begin{itemize}
\item PK: (codigo\_accion\_medica, razon social, nombre\_plan)

\item FK: 

o (codigo\_accion\_medica)

o (razon\_social,  nombre\_plan)

\item CC:

o (codigo\_accion\_medica, razon social, nombre\_plan)

\item ATRIBUTOS CON VALORES NULOS:  -

ACCION\_MEDICA(\emph{codigo\_accion\_medica}, unidades, condiciones\_paciente, 
descripción);
\end{itemize}

PLAN(\emph{razon\_social, }nombre\_plan);\label{HToc293405847}

\subsubsection{\textbf{Autorización}}

AUTORIZACIÓN(\emph{código\_accion\_medica, tipo\_doc\_profesional, numero\_doc\_profesional});

\begin{itemize}
\item PK: (codigo\_accion\_medica, tipo\_doc\_profesional, numero\_doc\_profesional)

\item FK: 

o (codigo\_accion\_medica)

o (tipo\_doc\_profesional, numero\_doc\_profesional)

\item CC:

o (codigo\_accion\_medica, tipo\_doc\_profesional, numero\_doc\_profesional)

\item ATRIBUTOS CON VALORES NULOS:  -.
\end{itemize}

ACCION\_MEDICA(\emph{codigo\_accion\_medica}, unidades, condiciones\_paciente, 
descripción);

PROFESIONAL(\emph{tipo\_doc\_profesional, numero\_doc\_profesional}, apellido\_nombre, 
email, número\_matrícula, celular\textit{\textbf{, }}codigo\_domicilio, codigo\_especialidad);\label{HToc293405848}

\subsubsection{\textbf{Realiza\_Accion}}

REALIZA\_ACCION(\emph{codigo\_accion\_medica, codigo\_especialidad, descripcion\_subespecialidad});

\begin{itemize}
\item PK: (codigo\_accion\_medica, codigo\_especialidad, descripcion\_subespecialidad)

\item FK: 

o (codigo\_accion\_medica)

o (codigo\_especialidad, descripcion\_subespecialidad)

\item CC:

o (codigo\_accion\_medica, codigo\_especialidad, descripcion\_subespecialidad)

\item ATRIBUTOS CON VALORES NULOS:  -.
\end{itemize}

ACCION\_MEDICA(\emph{codigo\_accion\_medica}, unidades, condiciones\_paciente, 
descripción);

SUBESPECIALIDAD(\emph{codigo\_especialidad, }descripcion\_subespecialidad);\label{HToc293405849}

\subsubsection{\textbf{Comprende}}

\textit{Diseño 1}

COMPRENDE(\emph{codigo\_turno} , código\_block\_horas);

\begin{itemize}
\item PK: (codigo\_turno)

\item FK: 

o (codigo turno)

o (codigo\_block\_horas)

\item CC:

o (codigo\_turno)

\item ATRIBUTOS CON VALORES NULOS:  -.
\end{itemize}

TURNO(\emph{código\_turno},fecha\_turno,hora\_turno,observaciones,lugar,sobreturno,bloqueado, 
código\_acción\_medica);

BLOCK\_HORAS(\emph{codigo\_block\_horas}, día, semana, hora\_desde, hora\_hasta, 
acepta\_sobreturnos, acepta\_fuera\_agenda, bloqueado, tipo\_turno, tipo\_agenda, 
cantidad\_pacientes, unidades,\emph{ }tipo\_doc\_profesional, numero\_doc\_profesional, 
codigo\_especialidad, descripcion\_subespecialidad);

\textit{Diseño 2}

TURNO(\emph{código\_turno}, fecha\_turno, hora\_turno,observaciones, lugar, sobreturno, 
bloqueado, código\_block\_horas, código\_acción\_medica);

BLOCK\_HORAS(\emph{codigo\_block\_horas}, día, semana, hora\_desde,hora\_hasta, 
acepta\_sobreturnos, acepta\_fuera\_agenda, bloqueado, tipo\_turno, tipo\_agenda, 
cantidad\_pacientes, unidades,\emph{ }tipo\_doc\_profesional, numero\_doc\_profesional, 
codigo\_especialidad, descripcion\_subespecialidad);

Se optó por el diseño 2 porque reduce el número de tablas.\label{HToc293405850}

\subsubsection{\textbf{Se realiza}}

\textit{Diseño 1}

SE\_REALIZA(\emph{código\_turno, }código\_accion\_medica);

\begin{itemize}
\item PK: (codigo\_turno)

\item FK: 

o (codigo\_accion\_medica)

o (codigo turno)

\item CC:

o (codigo\_turno)

\item ATRIBUTOS CON VALORES NULOS:-
\end{itemize}

TURNO(\emph{código\_turno}, fecha\_turno, hora\_turno, observaciones, lugar, sobreturno, 
bloqueado, código\_block\_horas);

ACCION\_MEDICA(\emph{código\_accion\_medica}, unidades, condiciones\_paciente, 
descripción);

\textit{Diseño 2}

TURNO(\emph{código\_turno}, fecha\_turno, hora\_turno, observaciones, lugar, sobreturno 
,bloqueado, código\_block\_horas, codigo\_accion\_medica);

ACCION\_MEDICA(\emph{código\_accion\_medica}, unidades, condiciones\_paciente, 
descripción);

Se opto por el diseño 2 porque reduce el número de tablas.\label{HToc293405851}

\subsubsection{\textbf{Define\label{HToc293405852}}}

\subsubsection{\textit{Diseño 1}}

DEFINE\_BLOCK\_HORAS (\emph{codigo\_block\_horas},  tipo\_doc\_profesional, numero\_doc\_profesional, 
codigo\_especialidad, descripcion\_subespecialidad);

\begin{itemize}
\item PK: (codigo\_block\_horas)

\item FK: 

o (tipo\_doc\_profesional, numero\_doc\_profesional, codigo\_especialidad, descripcion\_subespecialidad)

o (codigo\_block\_horas)

\item CC:

o (codigo\_block\_horas)

\item ATRIBUTOS CON VALORES NULOS:-
\end{itemize}

BLOCK\_HORAS(\emph{codigo\_block\_horas}, día, semana, hora\_desde,hora\_hasta, 
acepta\_sobreturnos, acepta\_fuera\_agenda, bloqueado, tipo\_turno, tipo\_agenda, 
cantidad\_pacientes, unidades);

TIENE\_ESPECIALIDAD(\emph{tipo\_doc\_profesional, numero\_doc\_profesional, codigo\_especialidad,} 
\emph{descripcion\_subespecialidad})

\textit{Diseño 2}

BLOCK\_HORAS(\emph{codigo\_block\_horas}, día, semana, hora\_desde,hora\_hasta, 
acepta\_sobreturnos, acepta\_fuera\_agenda, bloqueado, tipo\_turno, tipo\_agenda, 
cantidad\_pacientes, unidades,\emph{ }tipo\_doc\_profesional, numero\_doc\_profesional, 
codigo\_especialidad, descripcion\_subespecialidad);

TIENE\_ESPECIALIDAD(\emph{tipo\_doc\_profesional, numero\_doc\_profesional, codigo\_especialidad,} 
\emph{descripcion\_subespecialidad})

Se optó por el diseño 2 porque reduce el número de tablas.\label{HToc293405853}

\subsubsection{\textbf{Se atiende}}

SE\_ATIENDE(\emph{tipo\_doc\_paciente, numero\_doc\_paciente, código\_turno})

\begin{itemize}
\item PK: (tipo\_doc\_paciente, numero\_doc\_paciente, codigo\_turno)

\item FK: 

o (codigo\_turno)

o (tipo\_doc\_paciente, numero\_doc\_paciente)

\item CC:

o (tipo\_doc\_paciente, numero\_doc\_paciente, código\_turno)

\item ATRIBUTOS CON VALORES NULOS:  -.
\end{itemize}

TURNO(\emph{código\_turno}, fecha\_turno, hora\_turno, observaciones, lugar, sobreturno, 
bloqueado, código\_block\_horas,código\_acción\_medica);

PACIENTE(\emph{tipo\_doc\_paciente, numero\_doc\_paciente}, numero\_afiliado, fecha\_nacimiento, 
lugar\_nacimiento, tipo\_beneficiario, telefono, email, condicion\_iva, condicion\_paciente, 
numero\_historia\_clinica, apellido\_nombre, codigo\_domicilio, razon\_social, 
 nombre\_plan);\label{HToc293405854}

\subsubsection{\textbf{Turno\_Asignado}}

\textit{Diseño 1}

TURNO\_ASIGNADO(\emph{código\_turno}, tipo\_doc\_profesional, numero\_doc\_profesional);

\begin{itemize}
\item PK: (codigo\_turno)

\item FK: 

o (codigo\_turno)

o (tipo\_doc\_profesional, numero\_doc\_profesional)

\item CC:

o (codigo\_turno)

\item ATRIBUTOS CON VALORES NULOS: -
\end{itemize}

TURNO(\emph{código\_turno}, fecha\_turno, hora\_turno, observaciones, lugar, sobreturno, 
bloqueado, código\_block\_horas,código\_acción\_medica);

PROFESIONAL(\emph{tipo\_doc\_profesional, numero\_doc\_profesional}, apellido\_nombre, 
email, número\_matrícula, celular,\textit{\textbf{ }}codigo\_domicilio, codigo\_especialidad)

\textit{Diseño 2}

TURNO(\emph{código\_turno}, fecha\_turno, hora\_turno, observaciones, lugar, sobreturno, 
bloqueado, código\_block\_horas,código\_acción\_medica, tipo\_doc\_profesional, 
numero\_doc\_profesional)

PROFESIONAL(\emph{tipo\_doc\_profesional, numero\_doc\_profesional}, apellido\_nombre, 
email, número\_matrícula, celular,\textit{\textbf{ }}codigo\_domicilio, codigo\_especialidad)

Se optó por el diseño 1 porque facilita las búsquedas de turnos asignados a 
un profesional.\pagebreak{}\label{HToc293405855}

\newpage

\subsection{Scripts de creación}

%A continuación se incluyen los scripts de creación de las tablas del modelo
%relacional para ser ejecutado en un sistema de bases de datos que interprete
%SQL, particularmente las instrucciones DDL de dicho lenguaje. No se incluyen
%las instrucciones, normalmente específicas al motor propiamente dicho, de
%creación de usuarios, roles, schemas y bases de datos correspondientes.

\lstinputlisting[inputencoding=latin1,language=SQL,caption={Script de definición de la base de datos}]{../script/agenda_medica.sql}

\clearpage

\part{Apéndice}
\appendix

\section{Enunciado original}\label{sec:enunciado}
\includepdf[pages={-}, frame=true, pagecommand={}, noautoscale=true, scale=0.7]{doc/enunciado.pdf}

\clearpage

\section{\textbf{Forma de presentación del trabajo práctico}}

\begin{enumerate}
\item Presentar el diagrama de entidad - interrelación con indicaciones de restricciones 
de cardinalidad. 

\item Indicar dependencias de identidad y de existencia en el modelo. 

\item Especificar supuestos que justifiquen el modelo (Hipótesis). 

\item Presentar el diccionario de datos del diagrama con la siguiente información: 

Para cada tipo de entidad se debe especificar:

\begin{enumerate}
\item Definición. 

\item Especificación de atributos. 

\item Especificación de identificador único. 

Para cada tipo de interrelación se debe especificar: ~ 

\item Definición. 

\item Especificación de atributos. 

\item Especificación de identificador único. 
\end{enumerate}

\item Presentar el modelo Relacional ( \texttt{"}de tablas\texttt{"} )~ indicando 
para cada esquema de relación: 

\begin{enumerate}
\item Atributos 

\item Claves candidatas

\item Clave primaria

\item Claves foráneas 

\item Atributos que pueden tomar valores nulos

\item Realice el diagrama del Modelo de Tablas

\item Sentencias DDL
\end{enumerate}
\end{enumerate}

Nota: en los casos en que existan diferentes alternativas para efectuar la transformación 
de MER al modelo de tablas, elegir una única alternativa y enumerar las ventajas 
y desventajas de la alternativa elegida. \pagebreak{}\label{HToc293405800}

\end{document}
